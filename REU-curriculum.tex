% Options for packages loaded elsewhere
\PassOptionsToPackage{unicode}{hyperref}
\PassOptionsToPackage{hyphens}{url}
%
\documentclass[
]{article}
\usepackage{lmodern}
\usepackage{amssymb,amsmath}
\usepackage{ifxetex,ifluatex}
\ifnum 0\ifxetex 1\fi\ifluatex 1\fi=0 % if pdftex
  \usepackage[T1]{fontenc}
  \usepackage[utf8]{inputenc}
  \usepackage{textcomp} % provide euro and other symbols
\else % if luatex or xetex
  \usepackage{unicode-math}
  \defaultfontfeatures{Scale=MatchLowercase}
  \defaultfontfeatures[\rmfamily]{Ligatures=TeX,Scale=1}
\fi
% Use upquote if available, for straight quotes in verbatim environments
\IfFileExists{upquote.sty}{\usepackage{upquote}}{}
\IfFileExists{microtype.sty}{% use microtype if available
  \usepackage[]{microtype}
  \UseMicrotypeSet[protrusion]{basicmath} % disable protrusion for tt fonts
}{}
\makeatletter
\@ifundefined{KOMAClassName}{% if non-KOMA class
  \IfFileExists{parskip.sty}{%
    \usepackage{parskip}
  }{% else
    \setlength{\parindent}{0pt}
    \setlength{\parskip}{6pt plus 2pt minus 1pt}}
}{% if KOMA class
  \KOMAoptions{parskip=half}}
\makeatother
\usepackage{xcolor}
\IfFileExists{xurl.sty}{\usepackage{xurl}}{} % add URL line breaks if available
\IfFileExists{bookmark.sty}{\usepackage{bookmark}}{\usepackage{hyperref}}
\hypersetup{
  pdftitle={REU-Curriculum},
  pdfauthor={Yohn Jairo Parra Bautista, PhD. and Carlos Theran, PhD.},
  hidelinks,
  pdfcreator={LaTeX via pandoc}}
\urlstyle{same} % disable monospaced font for URLs
\usepackage[margin=1in]{geometry}
\usepackage{graphicx,grffile}
\makeatletter
\def\maxwidth{\ifdim\Gin@nat@width>\linewidth\linewidth\else\Gin@nat@width\fi}
\def\maxheight{\ifdim\Gin@nat@height>\textheight\textheight\else\Gin@nat@height\fi}
\makeatother
% Scale images if necessary, so that they will not overflow the page
% margins by default, and it is still possible to overwrite the defaults
% using explicit options in \includegraphics[width, height, ...]{}
\setkeys{Gin}{width=\maxwidth,height=\maxheight,keepaspectratio}
% Set default figure placement to htbp
\makeatletter
\def\fps@figure{htbp}
\makeatother
\setlength{\emergencystretch}{3em} % prevent overfull lines
\providecommand{\tightlist}{%
  \setlength{\itemsep}{0pt}\setlength{\parskip}{0pt}}
\setcounter{secnumdepth}{-\maxdimen} % remove section numbering

\title{REU-Curriculum}
\author{Yohn Jairo Parra Bautista, PhD. and Carlos Theran, PhD.}
\date{5/14/2021}

\begin{document}
\maketitle

\hypertarget{hands-on-research-data-science-and-machine-learning-applications-for-images-and-behavior-analysis}{%
\subsection{Hands-On Research: Data Science and Machine Learning
Applications for Images and Behavior
Analysis}\label{hands-on-research-data-science-and-machine-learning-applications-for-images-and-behavior-analysis}}

\hypertarget{models-and-methods-for-image-and-text-data-with-ml.}{%
\subsubsection{Models and Methods for Image and Text data with
ML.}\label{models-and-methods-for-image-and-text-data-with-ml.}}

Why will it be interesting to have you enroll in this summer camp? Would
you like to increase your knowledge and creativity using science,
technology, mathematics, and you want to know the models used by the
most prominent companies to track your interests. In this Hands-On
Research for Undergraduates program, we will deliver the main concepts
that combine statistical and computational theory to create and train AI
(Machine Learning) models for classification and prediction problems in
the field of image and Behavior analysis. Hands-On Research will take
place on ------------- (8 weeks).

The summer camp will provide experience in \textbf{formulating} and
carrying out a tangible \textbf{data science analysis} with real-world
data. The capstone will be a \textbf{group/team project (3/5 students)}
and each project will focus in open, pre-existing secondary data.

Students will be computing with data using popular language like
\emph{Python and R}, but also specific languages for transforming and
manipulating text, and for managing complex computational pipelines.

One of the outcomes of the summer project will be to develop a
\textbf{package that abstract commonly used pieces of workflow} and make
them available for use in \textbf{future projects}.

The student will learn how to analyze and interpret geospatial images to
classify the diverse types of vegetation presented in the image.

\end{document}
